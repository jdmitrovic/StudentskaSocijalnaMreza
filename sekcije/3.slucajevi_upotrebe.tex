\section{Slučajevi upotrebe}

\subsection{Pristupanje mreži}
\subsubsection{Kreiranje korisničkog profila}
\begin{itemize}
\item Kratak opis - Korisnik se registruje tako što popunjava formu svojim ličnim podacima. Nakon validacije unetih podataka korisnik prelazi na sledeći korak. Kako bi upotpunio kreiranje profila bira oblasti interesovanja na osnovu kojih će mu objave biti filtrirane.
\item Učesnici - neregistrovan korisnik
\item Preduslovi - korisnik ima pristup internetu
\item Postuslovi - korisniku je kreiran nalog
\item Glavni tok
	\begin{enumerate}
	\item Korisnik odlazi na stranicu za registrovanje novih korisnika
	\item Korisnik unosi lične podatke i klikne na dugme "Sledeći korak"
	\item Vrši se validacija podataka
	\item Na sledećoj strani korisnik bira oblasti interesovanja prema kojim će mu se prikazivati objave (bira iz padajuće liste sa opcijom pretraživanja)
	\item Korisnik klikne na dugme "Završi registraciju"
	\item Sistem privremeno čuva unete podatke
	\item Korisniku je stigao aktivacioni e-mail sa linkom za potvrdu registracije
	\item Korisnik otvara mail i klikne na link za potvrdu registracije
	\item Sistem aktivira nalog
	\item Korisnik dobija obaveštenje da je nalog kreiran
	\end{enumerate}
\item Alternativni tokovi
	\begin{itemize}
	\item[3.a] Nevalidna prijava - sistem ukazuje korisniku na kom polju forme podaci nisu korektno uneti. Slučaj upotrebe se nastavlja na koraku 2.
	\item[7.a] Korisnik nije potvrdio registraciju u odredjenom vremenskom intervalu - link za potvrdu je istekao. Slučaj upotrebe se završava
	\end{itemize}
\end{itemize}

\subsubsection{Kreiranje profila organizacija}
\begin{itemize}
\item Kratak opis - Organizacije pristupaju mreži tako što popunjavaju formu odgovarajućim podacima. Tu formu proverava administrator nakon čega odobrava profil. Time organizacija postaje ravnopravan član mreže.
\item Učesnici - organizacija i administrator
\item Preduslovi
	\begin{itemize}
	\item Organizacija je validno pravno lice koje može pristupiti mreži 
	\item Organizacija ima pristup internetu
	\end{itemize}
\item Postuslovi - organizaciji je kreiran nalog
\item Glavni tok
	\begin{enumerate}
	\item Organizacija odlazi na stranicu za registrovanje organizacija
	\item Organizacija unosi podatke i klikne na dugme "Završi unos podataka"
	\item Sistem proverava unete podatke
	\item Sistem obaveštava organizaciju o uspešnosti inicijalne provere podataka i daje mogućnost korigovanja unetih podataka.
	\item Organizacija klikne na dugme "Završi registraciju"
	\item Uneti podaci se šalju administratoru na pregled
	\item Administrator odobrava pristup organizacije socijalnoj mreži i šalje e-mail za potvrdu naloga
	\item Organizacija potvrđuje nalog korišćenjem linka poslatog e-mailom
	\item Sistem aktivira nalog
	\item Organizacija dobija obaveštenje da je nalog kreiran
	\end{enumerate}
\item Alternativni tokovi
	\begin{itemize}
	\item[3.a] Uneti podaci nisu ispravni - sistem ukazuje koji podaci nisu adekvatni. Slučaj upotrebe se nastavlja na koraku 2.
	\item[7.a] Organizacija nije validan korisnik mreže. Administrator šalje e-mail kojim obaveštava organizaciju da je njihov zahtev za pristup mreži odbijen. Slučaj upotrebe se završava.
	\item[8.a] Link za potvrdu naloga je istekao. Slučaj upotrebe se završava.
	\end{itemize}
\end{itemize}

\subsection{Ažuriranje profila}
\begin{itemize}
\item Učesnici - korisnik, organizacija
\item Preduslovi - korisnik je registrovan (organizacija se pridružila mreži)
\item Postuslovi - izmenjeni su željeni podaci
\item Glavni tok
\begin{enumerate}
\item Korisnik (organizacija) se prijavljuje na sistem
\item Korisnik (organizacija) odlazi na stranicu za ažuriranje podataka klikom na opciju ,,Izmeni profil"
\item Sistem prikazuje formu koja sadrži podatke koji se mogu menjati
\item Korisnik (organizacija) pravi željene izmene
\item Klikom na opciju ,,Sačuvaj" potvrdjuje izmene
\item Sistem ažurira napravljene izmene
\end{enumerate}
\end{itemize}

\subsection{Kreiranje korisničkih grupa}
\begin{itemize}
\item Korisnici aplikacije mogu kreirati grupu koja je jedinstveno određena nazivom. 
\item Prilikom kreiranja korisnik bira da li će grupa biti:

\begin{enumerate}
\item vidljiva - mogu je posetiti i pretražiti svi korisnici mreže
\item sakrivena - mogu je posetiti i pretražiti samo clanovi iste
\end{enumerate}

\item	Kreator grupe poziva druge korisnike ili ogranizacije da se učlane.
\item	Kreator u svakom trenutku može da obriše člana ili bilo koju objavu ako smatra da je neprikladna.
\item	Pored objava mogu se praviti i eventovi tematski slični grupi.
\end{itemize}

\subsection{Deljenje sadržaja}
\begin{itemize}
\item Kratak opis - Korisnici mreže mogu kreirati objave koje će biti prikazane odabranoj publici. Odabir publike vrši sam kreator pažljivim odabirom oblasti na koje se objava odnosi. Takođe, mogu se targetirati i studenti određenih fakulteta.
\item Učesnici - korisnici mreže
\item Preduslovi - učesnici su registrovani
\item Postuslovi - kreriana objava vidljiva je na profilu korsnika i na početnim stranama targetiranih korisnika
\item Glavni tok
    \begin{enumerate}
	\item Korisnik klikne na dugme "Kreiraj novu objavu" nakon čega mu se prikazuju polja za unos
	\item Bira tip objave: obaveštenje, pitanje, tutorijal ili ponuda
	\item Popunjava polja karakteristična za tip objave (podtok)
	\item Unosi tekst objave gde može i da označi druge korisnike
    \item Bira oblasti na koje se objava odnosi
    \item Može da izabere opciju za vidljivost objave
    \item Može da izabere odredjene fakultete ukoliko postoji povezanost sa sadržajem
    \item Nakon klika na dugme "Objavi", sistem obrađuje podatke, upisuje u bazu i objava je kreirana
    \end{enumerate}
\item Podtok - popunjavanje polja karakterističnih za tip objave
    \begin{itemize}
        \item Ukoliko je izabran "tutorijal"
            \begin{enumerate}
            \item Kreator mora da učita fajl
            \item Mora da izabere folder ili kreira novi radi organizacije tutorijala
            \item Može da označi saradnike na tutorijalu
            \end{enumerate}
         \item Ukoliko je izabraa "ponuda"
            \begin{enumerate}
            \item Kreator mora da navede tip ponude (npr rad na projektu, ponuda za posao/praksu, držanje časova, i slično)
            \end{enumerate}    
	 \end{itemize}
\end{itemize}

\subsection{Pretraživanje}
\begin{itemize}
\item Kratak opis - Korisnici i organizacije mogu vršiti pretragu po odredjenom kriterijumu i odabrati da li žele da pronadju objave, druge korisnike, organizacije ili grupe koji sadrže u nazivu (sadržaju u slucaju objava) ključne reci po kojim se pretražuje. Kriterijumi pretrage (filteri) se mogu kombinovati. Omogucava se lakše pronalaženje sadržaja koji interesuju korisnike mreže.
\item Učesnici - Korisnici, organizacije
\item Preduslovi - Korisnik (organizacija) - registrovani
\item Postuslovi - Prikazani su svi korisnici/organizacije/grupe/objave (u zavisnosti od toga sta se pretražuje) koje zadovoljavaju kriterijum pretrage 
\item Glavni tok
	\begin{enumerate}
	\item Korisnik (organizacija) unosi u polje za pretragu ključnu rec(ili vise njih) i klikne na dugme "Pretraga"
	\item Izabere se predmet interesovanja - objava, korisnik, organizacija, grupa
	\item Prikazuje se : u slučaju objava - sve objave koje u svom sadržaju imaju unete filtere, u slučaju korisnika, organizacija, grupa - svi koji u svom nazivu imaju unete filtere
	\end{enumerate}
\item Alternativni tok
	\begin{enumerate}
	\item Nije pronadjen predmet interesovanja
	\item Sve dok korisnik (organizacija) želi da vrši pretragu, poništava filtere pretrage i unosi nove 
	\end{enumerate}
\end{itemize}


\subsection{Kontrolisanje sadržaja i profila}
\begin{itemize}
\item Kratak opis - Administrator vrši kontrolu nad profilima korisnika (organizacija), kao i kontrolu objavljenog sadržaja, kako ne bi došlo do zloupotreba mreže ili deljenja neželjenog sadržaja. 
\item Učesnici - Administrator
\item Preduslovi - Korisnik (organizacija) - registrovani
\item Postuslovi - Sadržaj koji je objavljen na mreži od strane korisnika (organizacija) je u skladu sa politikom socijalne mreze;
informacije na profilima korisnika (organizacija) su ispravne i u skladu sa pravilima mreže
\item Glavni tok
	\begin{itemize}
	\item Slucaj kontrole profila
	\begin{enumerate}
	\item Administrator dobija informaciju o sumnjivom podacima na profilu korisnika (stranici organizacije)
	\item Kako je administratorima omogucen pristup profilima, vrši proveru informacija
	\item Ukoliko se ispostavi da informacije nisu ispravne ili nisu u skladu sa pravilima mreže, administrator kontaktira korisnika (organizaciju) radi dalje provere i nakon toga, ukoliko je potrebno, uklanja profil
	\end{enumerate}
	\end{itemize}
	\begin{itemize}
	\item Slucaj kontrole sadržaja
	\begin{enumerate}
	\item Administrator dobija informaciju o sadržaju podeljenom na mreži koji nije u skladu sa politikom mreže
	\item Nakon provere objave, administrator je uklanja ukoliko je neželjenog sadržaja 
	\item Administrator zatim izvrši kontrolu profila korisnika (organizacije) koji je podelio objavu (opisano u prethodnom koraku)
	\end{enumerate}
	\end{itemize}
\end{itemize}
